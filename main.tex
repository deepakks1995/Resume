%-------------------------
% Resume in Latex
% Author : Deepak Sharma
%------------------------

\documentclass[letterpaper,11pt]{article}
\usepackage{fontawesome}
\usepackage{latexsym}
\usepackage[empty]{fullpage}
\usepackage{titlesec}
\usepackage{marvosym}
\usepackage[usenames,dvipsnames]{color}
\usepackage{verbatim}
\usepackage{enumitem}
\usepackage[pdftex]{hyperref}
\usepackage{fancyhdr}


\pagestyle{fancy}
\fancyhf{} % clear all header and footer fields
\fancyfoot{}
\renewcommand{\headrulewidth}{0pt}
\renewcommand{\footrulewidth}{0pt}

% Adjust margins
\addtolength{\oddsidemargin}{-0.375in}
\addtolength{\evensidemargin}{-0.375in}
\addtolength{\textwidth}{1in}
\addtolength{\topmargin}{-.5in}
\addtolength{\textheight}{1.0in}

\urlstyle{same}

\raggedbottom
\raggedright
\setlength{\tabcolsep}{0in}

% Sections formatting
\titleformat{\section}{
  \vspace{-4pt}\scshape\raggedright\large
}{}{0em}{}[\color{black}\titlerule \vspace{-5pt}]

%-------------------------
% Custom commands
\newcommand{\resumeItem}[2]{
  \item\small{
    \textbf{#1}{: #2 \vspace{-2pt}}
  }
}

\newcommand{\resumeNewItem}[3]{
  \item\small{
    \textbf{#1:}{\hfill {#2}\\}{ #3\vspace{-2pt}}
  }
}

\newcommand{\resumeSubheading}[4]{
  \vspace{-1pt}\item
    \begin{tabular*}{0.97\textwidth}{l@{\extracolsep{\fill}}r}
      \textbf{#1} & #2 \\
      \textit{\small#3} & \textit{\small #4} \\
    \end{tabular*}\vspace{-5pt}
}

\newcommand{\resumeSubItem}[2]{\resumeItem{#1}{#2}\vspace{-4pt}}
\newcommand{\resumeNewSubItem}[3]{\resumeNewItem{#1}{#2}{#3}\vspace{-4pt}}

\renewcommand{\labelitemii}{$\circ$}

\newcommand{\resumeSubHeadingListStart}{\begin{itemize}[leftmargin=*]}
\newcommand{\resumeSubHeadingListEnd}{\end{itemize}}
\newcommand{\resumeItemListStart}{\begin{itemize}}
\newcommand{\resumeItemListEnd}{\end{itemize}\vspace{-5pt}}

%-------------------------------------------
%%%%%%  CV STARTS HERE  %%%%%%%%%%%%%%%%%%%%%%%%%%%%


\begin{document}

%----------HEADING-----------------
\begin{tabular*}{\textwidth}{l@{\extracolsep{\fill}}r}
  \textbf{{\Large {Deepak Sharma}}} \\
  B.Tech, Computer Science and Engineering \\
  Indian Institute of Technology Mandi \\
  \href{mailto:deepakks1995@gmail.com}{deepakks1995@gmail.com} \\
  \faGithub\href{https://github.com/deepakks1995}{ deepakks1995} \\
    \faLinkedin\href{https://www.linkedin.com/in/deepakks1995}{ deepakks1995} \\ 
  +91-8894054791
\end{tabular*}

%-----------EXPERIENCE-----------------
\section{Work Experience}
  \resumeSubHeadingListStart
    
    \resumeSubheading
      {Khosla Labs}{Bangalore, India}
      {Associate Software Engineer}{July 2018 - Present}
      \resumeItemListStart
        \item {
            Lead the ML team to build classifiers (with limited data), train Object-Detection models like Mask-RCNN, Faster-RCNN, build OCR parser for Google Vision Api supporting multiple documents.
        }
        \item {
            Converted tensorflow models like Mask-RCNN, Faster-RCNN, Mobilenet to servables(accepting base64 images), later deploying them using tensorflow serving.
        }
        \item {
            Build a Document scanner using OpenCV to detect the edges of document while scanning. 
        }
        \item {
            Used OpenCv to implement OpenCv's MSER algorithm for getting bounding boxes around texts, later build a python library to add noise around an ID image also supporting random background addition(training Object Detection), rotation, cropping around the edges.
        }
        
        \item {
            Worked in a team handling a set of multi-tenant microservices using MySQL. Developed new microservices,
            enhanced database schemas and testing frameworks to accommodate new use cases and built APIs.
        }
        \item {
            Developed a web portal using Angular 6 for client onboarding.
        }
        \item {
            Developed and managed Api's and android app for a new product, requiring QR Code reading both from camera as well as from scanned Pdf, reading compressed xml data and verifying its signature.
        }
      
    %   \item {
    %         Worked with the data migration team to map older databases to newer ones, and then to redirect monolith api's to their corresponding microservices.
    %     }
        % \item {
        %     Worked on proprietary framework, built on spring boot for api development.
        % }
        % \item {
        %     Integrated the developed api's with the testing framework, also extended the framework to support functional testing using mockito. Created and executed unit tests and performed basic api testing.
        % }
        
        
      \resumeItemListEnd
    
    \resumeSubheading
      {Khosla Labs}{Bangalore, India}
      {Internship}{Dec 2016 - Feb 2017}
      \resumeItemListStart
        \item {
            Developed a dashboard named 'Data-Juno Monitor' on top of Angular JS framework to feed live data into different visualizations
        }
        \item {
            Used D3js and Rickshaw JS(an open-source JavaScript toolkit) for implementing interative visualizations like Tree Map, Pie Charts, Forced Directed Graphs etc.
        }
        % \item
        %   {My work also included writing Android Background Services to fetch data from the client side}
      \resumeItemListEnd
  \resumeSubHeadingListEnd

%-----------PROJECTS-----------------
\section{Projects}
  \resumeSubHeadingListStart
    \resumeNewSubItem{Major Technical Project}{(Aug 2017 - May 2018), IIT Mandi}{
      The project aimed at creating a neural network that implements deep hashing and convert an image into a low dimensional binary vector (hash). Datasets consists of bio-metric images like fingerprints, iris, knuckle images etc. Challenges include:
      \begin{itemize}
        \item Handling alignment, translation of images.
        \item Minimizing quantization loss of obtained binary vectors.
        \item Propose an efficient indexing problem. 
      \end{itemize}
      Implemented \href{https://arxiv.org/abs/1801.08360}{DADH} network, as to get the distance between the binary encoding of the bio-metric images. Images were preprocessed using an Auto Encoder which was trained on FVC2002, FVC2004 and FVC2006 datasets.
      \linebreak
    }

    \resumeNewSubItem{FPSensorNet-A Deep CNN for Fingerprint Sensors Classification}{(Mar 2017 - May 2017), IIT Mandi}{
        \begin{itemize} 
          \item Classify fingerprint images on the basis of Sensors
          \item Resnet50 layer network vs VGG16 was used to extract quality features of images.
          \item Resnet50 layers were pruned and unnecessary layers were removed by applying a dropout 
          \item Added various factors such as Noise, Occlusion, Rotation to test the efficiency of network
          % \item Paper "FPSensorNet: A Deep CNN for Fingerprint Sensors Classification" submitted to WIFS 2017.
        \end{itemize}
    }
    \resumeNewSubItem{On the Fly Encryptor}{ VI Semester, IIT Mandi}{
        Modified Linux Kernel 4.1.0 to implement a file encryption mechanism without affecting user experience. 
        \begin{itemize} 
          \item Sticky bit (A permission bit) was used to tag a file whether or not encrypted.
          \item Read and write functions vfs$_{\_}$read() \& vfs$_{\_}$write()  at kernel level were tweaked accordingly.
           \end{itemize}
        %   \item Later Inodes were modified to create extra space for encryption.
       
    }
    % \resumeNewSubItem{Self-Arranging Chairs}{IV Semester, IIT Mandi}{
    %     \begin{itemize} 
    %       \item Raspberry Pi controlled module which brings a chair back to its original position. 
    %       \item Used Raspbian Python Code to fetch raw data from optical sensor and backtracking the traced path. 
    %       \item Handling multiple input streams from optical sensor, sound, obstacle sensors while generating output for DC geared motors to retrace path. 
    %       \item Project was inspired from Nissan Parking Chair Concept - \href{https://www.youtube.com/watch?v=O1D07dTILH0}{YouTube Link} 
    %     \end{itemize}
    % }
    % \resumeNewSubItem{Institute Transport Portal}{III Semester, IIT Mandi}{
    %   Part of course Applied Database Practicum. The project was for our Institute Transport System, it served the following purposes:
    %   \begin{itemize}
    %     \item  Displaying bus timetable 
    %     \item Facility to book a vehicle
    %     \item Making Complaints
    %     \item Tracking a Bus
    %   \end{itemize}
    % Background: LAMP Stack, PHP, MySQL, Bootstrap, HTML, CSS
    % }

    % \resumeNewSubItem{Online Banking System}{III Semester, IIT Mandi} {
    %   Part of course Applied Database Practicum. The portal provides the facility for online secure money transactions, creating accounts, checking balance and a robust database for the employee side. \\
    %   Programming Background: MySQL, PHP, JavaScript, HTML, CSS, LAMP
    % }

    % \resumeNewSubItem{Attendance Machine}{II Semester, IIT Mandi}{
    %   \begin{itemize} 
    %     \item The project was to mark attendance using ID-Cards
    %     \item ID-Cards contains Roll numbers of students in binary format.
    %     \item Used Embedded C and Arduino-Mega to mark attendance.
    %   \end{itemize}
    % }

    % \resumeNewSubItem{Line Follower Robot}{I Semester, IIT Mandi}{
    %   \begin{itemize} 
    %     \item Used Embedded C and AVR Studio to program an AtMega16 microcontroller 
    %   \end{itemize}
    % }

  \resumeSubHeadingListEnd


% %-----------COURSE WORK-----------------
% \section{Course Work}
%   \resumeSubHeadingListStart
%   \begin{itemize}
%   	\begin{minipage}{0.5\linewidth}
%         \item Computation for Engineers
%         \item Advanced Data Structures and Algorithms
%         \item Information and Database Systems
%         \item Communication and Distributing Process 
%         \item Deep Learning and Application
%   	\end{minipage}
%   	\begin{minipage}{0.45\linewidth}
%         \item System Practicum
%         \item Pattern Recognition 
%         \item Artificial Intelligence
%         \item Mathematical Foundations of Computer Science 
%         \item Computer Organization
%   	\end{minipage}
%   \end{itemize}
%   \resumeSubHeadingListEnd

%-----------EDUCATION-----------------
\section{Education}
  \resumeSubHeadingListStart
    \resumeSubheading
      {Indian Institute of Technology Mandi}{Mandi, Himachal Pradesh}
      {B.Tech in Computer Science and Engineering;  CGPA: 7.17}{Aug. 2014 -- May. 2018}
    \resumeSubheading
      {Kapil Gyanpeeth}{Mansarover, Jaipur}
      {Higher Secondary;  Percentage: 91.8\% }{2013}
    \resumeSubheading
      {Kapil Gyanpeeth}{Mansarover, Jaipur}
      {Senior Secondary;  CGPA: 7.8 }{2011}
  \resumeSubHeadingListEnd
  
% --------PROGRAMMING SKILLS------------
\section{Programming Skills}
 \resumeSubHeadingListStart
    \item{
     \textbf{Programming Languages}{: Java, Python, C++}
     \hfill
    }
    \item{
      \textbf{Python Packages}{: Tensorflow, Keras, OpenCV}
      \hfill
    }
    \item{
       \textbf{Web}{: Angular 6, RxJS}
      \hfill
    }
    \item{
      \textbf{Software Packages}{: Android Studio, Intellij Idea Community Edition}
      \hfill
    }
    \item{
      \textbf{Platforms}{:Mac and Linux}
      \hfill
    }
    \item{
    \textbf{Others}{: Docker, Tensorflow Serving}
   }
 \resumeSubHeadingListEnd






%-----------SCHOLASTIC ACHIEVEMENTS-----------------
% \section{Scholastic Achievement}
%   \resumeSubHeadingListStart 
%   \begin{itemize}
%     \item Selected for the Pre-Elimination Round of Snack Down, 2016\\ (CodeChef's annual multi-round programming competition)
%     \item ACM-ICPC India Regionals, Online Preliminary Round 2016 – Rank 1179, Team SPDCode
%     \item Google APAC 2017 University Test Round E – Rank 1177, Handle deepakks19995
%     \item JEE Advanced AIR – 2486 (Gen, CML)
%   \end{itemize}
%   \resumeSubHeadingListEnd


%-------------------------------------------
\end{document}